\documentclass[10pt,a4paper]{article}
\usepackage[
nochapters, % Turn off chapters since this is an article        
beramono, % Use the Bera Mono font for monospaced text (\texttt)
eulermath,% Use the Euler font for mathematics
pdfspacing, % Makes use of pdftex’ letter spacing capabilities via the microtype package
dottedtoc % Dotted lines leading to the page numbers in the table of contents
]{classicthesis} 
\usepackage[a4paper,top=2.5cm,bottom=3cm,left=2.8cm,right=2.8cm,bindingoffset=0mm]{geometry}
\usepackage[utf8]{inputenc}
\usepackage[T1]{fontenc}
\usepackage[italian]{babel}
\usepackage{amsmath}
\usepackage{amsfonts}
\usepackage{amssymb}
\usepackage{makeidx}
\usepackage{placeins}
\usepackage{graphicx}
\makeatother
\author{Fabio Prestipino}
\title{Guida ROOT}
\begin{document}
	
	\maketitle
	\tableofcontents
	\newpage
\section{Introduzione ed informazioni generali utili}
ROOT è un linguaggio ad oggetti sviluppato dal CERN basato su C++ utile per l'analisi ed elaborazione di grandi moli di dati. Ogni istogramma, grafico, funzione (...) che creiamo è un oggetto con delle sue opzioni e funzioni (funzioni membro). Una proprietà fondamentale è l'ereditarietà: tutte le classi sono figlie di una classe di base (TObject) da cui discendono le altre; ogni classe figlia eredita le funzioni membro di quelle precedenti. Il tipo degli oggetti comincia sempre con la lettera T, seguita da altre lettere che specificano le caratteristiche dell'oggetto. Di seguito riporto alcune informazioni di base in ordine casuale:
\begin{itemize}
\item I tipi base di ROOT hanno nomi leggermente diversi da quelli di c++: \(Int\_t\), \(Float\_t\), \(Double\_t\), TString...\\
\item Per stampare a schermo basta usare i comandi di C++. Ad esempio:\\\\
Double\_t a;\\
cout<< "Scritta" << a;\\
\item Per accedere ai file di ROOT il codice da scrivere sulla barra di comando è\\\\
TBrowser b;\\
\item Per usare funzioni matematiche è possibile utilizzare la libreria di Root "TMath", ad esempio:\\\\
TMath::Sin(TMath::Pi()/2.) = 0\\\\ 
\item Per salvare un oggetto (grafico, istogramma,...) come file ROOT bisogna aprire un file ROOT scriverci l'oggetto dentro e chiuderlo:\\\\
TFile *nome = new TFile(“nome\_file.root”,”RECREATE”);\\
nome\_oggetto -> Write();\\
nome -> Close();\\
\item Per accedere a file di dati (che dovrà essere nella stessa cartella della macro) si usa la funzione di C++ fstream(); in ordine bisogna creare un oggetto file, aprire quello desiderato, accedervi e chiuderlo. In ordine:\\\\
 fstream nome;\\
 nome.open("nome\_file.txt");\\
 nome >> x;\\
 nome.close();\\\\
\end{itemize}
\subsection{Come interfacciarsi}
Esistono 3 modi di interfacciarsi a ROOT: 
\begin{itemize}
\item \textbf{Command line}\\
Utile per comandi di base, è più flessibile delle macro (ad esempio non serve il ;). Per accedere al terminale direttamente da ROOT basta scrivere i normali comandi del terminale preceduti da un ".! ". Per uscire da ROOT basta scrivere sulla barra di comando .q oppure se non funziona .qqqqq...
\item \textbf{Macros}\\
Un file scritto in un editor di testo che poi viene eseguito su ROOT. Una macro si presenta come una normale funzione di c++, può prendere parametri che poi vengono usati al suo interno e si possono creare altre funzioni all'interno. Di seguito un esempio di macro vuota\\\\
void nome(parametro1, parametro 2, ...)\{\}\\\\
\'{E} inoltre possibile far girare una macro in due modi diversi: o facciamo partire tutta la macro, eseguendo così tutte le funzioni al suo interno, con il comando\\\\
.c name.C\\\\
oppure carichiamo le funzioni della macro in memoria con il comando\\\\
.L name.C\\\\
per poi chiamare le funzioni singolarmente, ad esempio:\\\\
funz1();\\\\
IMPORTANTE: LE MACROS DEVONO STARE NELLA CARTELLA "HOME\$/root/macros"
\item \textbf{Finestra grafica (GUI)}\\
Per curare l'estetica dei grafici è possibile sfruttare l'interfaccia grafica, più intuitiva. Per far comparire le opzioni della finestra bisogna cliccare su "view" e spuntare "editor", "toolbar" ed "event statusbar". 
\end{itemize}
\section{Istogrammi}
\subsection{Inizializzare e riempire un'istogramma}
Il tipo degli istogrammi è formato da 4 lettere: la T(come per tutti gli oggetti), una H di "histogram", un numero che indica il numero di dimensioni dell'istogramma e un'ultima lettera che può essere I (int), F (float), D (double) che indica il modo in cui vengono rappresentati i dati nell'istogramma. Un'istogramma di base sarà, ad esempio, del tipo "TH1F".\\
\'{E} raccomandato dichiarare (costruire) un'istogramma usando i puntatori, di seguito un format di dichiarazione di un'istogramma unidimensionale:\\\\
TH1F *nome = new TH1F("nome", "titolo", n-bin, \(x_{min}\), \(x_{max}\)); \\\\
Dove n-bin è il numero di bin che vogliamo nell'istogramma ed \(x_{min}\), \(x_{max}\) il range entro cui vogliamo vedere l'istogramma. Per aggiungere punti ad un'istogramma unidimensionale:\\\\
nome -> Fill(x);\\\\
la funzione Fill può accettare anche un secondo paramentro che permette di aggiungere in una volta tante entrate uguali:\\\\
nome -> Fill(x, 8);\\\\
aggiunge 8 entrate x all'istogramma.\\
Se invece volessimo più dimensioni:\\\\
TH2F *nome = new TH2F("nome", "titolo", \(n_x-bin\), \(x_{min}\), \(x_{max}\), \(n_y-bin\), \(y_{min}\), \(y_{max}\)); \\\\
Per aggiungere punti in un'istogramma bidimenzionale:\\\\
nome -> Fill(x,y);\\\\
\'{E} inoltre possibile fare le proiezioni degli istogrammi multidimenzionali sui singoli assi:\\\\
nome1 -> ProjectionX();\\
nome2 -> ProjectionY();\\\\
che crea un'istogramma già riempito.\\

\subsection{Inserire dati da file}
\'{E} possibile riempire un'istogramma sfruttando i dati di un file di testo (che dovrà essere nella stessa cartella della macro). Innanzitutto bisogna costruire l'oggetto file\\\\
fstream nome;\\\\
poi bisogna aprire il file desiderato\\\\
nome.open("nome\_file.txt")\\\\
infine con un ciclo aggiungiamo i dati del file fin quando questo non finisce:\\\\
Float\_t x;\\
while(1)\{\\
	nome >> x;\\
	nome\_isto -> Fill(x);\\
	if(!nome.good()) break;\\
\}\\
nome.close();\\\\

Dove la funzione nome.good() dice se il file è concluso o meno. Nell'ultima riga è stato chiuso il file. 
\subsection{Metodi utili istogrammi}
Di seguito sono riportati i metodi più importanti per gli istogrammi
\begin{itemize}
	\item nome ->GetXaxis()->SetTitle("nome");
	setta il titolo dell'asse
	\item nome -> SetMarkerStyle(numero);\\
	assegna uno stile al marker
	\item nome -> SetMarkerSize(numero);\\
	assegna una grandezza al marker
	\item nome -> GetBinError(nbin);\\
	prende l'errore di un bin
	\item nome -> GetBinContent(nbin);
	prende il numero di entrate nel bin
	\item nome -> GetBinContent(0);
	prende il numero di entrate in underflow
	\item nome -> GetBinContent(nbin+1);
	prende il numero di entrate in overflow
	\item nome -> GetEntries();\\
	prende il numero di entrate totali
	\item nome -> SetBinContent(ibin,val);\\
	assegna al contenuto del bin un valore val
	\item nome -> GetMean();\\
	calcola la media
	\item nome -> GetRMSError();\\
	calcola l'errore sulla media
	\item nome -> GetRMS();\\
	calcola la deviazione standard
	\item nome -> GetMeanError();\\
	calcola l'errore sulla deviazione standard
	\item nome -> GetMaximum();\\
	prende il bin con numero massimo di entrate
	\item nome -> GetMaximumbin();\\
	prende il numero del bin con numero massimo di entrate
	\item nome -> GetBinCenter();\\
	prende il centro del bin
	\item nome -> GetIntegral();\\
	restituisce un'array dei conteggi cumulativi 
	\item nome -> Integral();\\
	calcola l'integrale nel range
\end{itemize}
\subsection{Operazioni fra istogrammi}
Si possono usare metodi dedicati della classe per effettuare operazioni tra oggetti di ROOT; le operazioni utili sono somma (talvolta sottrazione) e divisione. Seguiamo gli step necessari:
\begin{enumerate}
	\item Creare e riempire gli istogrammi con cui operare
	\item Creare l'istogramma risultato, inizialmente vuoto, passando come parametro il primo istogramma con cui si vole operare. Per la divisione:\\\\
	TH1F *nome\_divisione = new TH1F(*h1);\\
	nome\_divisione -> Divide(h1,h2,1,1);\\\\
	Per la somma:\\\\
	TH1F *nome\_somma =new TH1F(*h1);\\
	nome\_somma -> Add(h1,h2,1,1);\\\\
	Per la sottrazione si usa la stessa sintassi dell'addizione ma con un "-1" al posto del secondo "1". Infine, per avere incertezze corrette sugli istogrammi risultato, è necessario usare il metodo Sumw2():\\\\
	nome\_risultato -> Sumw2();\\\\
\end{enumerate}
\section{Grafici}
Un grafico rappresenta una serie di N coppie di due variabili (x, y). Esistono due tipi di grafici: quello semplice TGraph e quello in cui è possibile includere anche gli errori nel grafico TGraphErrors. 
\subsection{Costruttore di TGraph}
Il costruttore di TGraph di base genera semplicemente un grafico di n punti presi da due array (una che contiene tutte le x ed una le y). \\\\
TGraph *nome = new TGraph (numero\_punti, *nome\_arrayx, *nome\_arrayy)\\\\
dove le array sono passate con il puntatore (nome preceduto da asterisco).\\
Un altro importante costruttore di TGraph è quello che prende i punti da un file esterno, la sua inizializzazione è \\\\
TGraph *nome = new TGraph (*filename, *format="\%lg \%lg") \\\\
dove il format fornisce informazioni al programma su come prendere i dati dal file di testo, in questo caso "\%lg \%lg" dice di prendere le prime due colonne di numeri double separate da un blank delimiter (spazio bianco). Se si volesse saltare una colonna basta aggiungere un'asterisco: "\%lg \%*lg \%lg". Se si inseriscono più di due colonne (al minimo devono essere due), si crea un grafico a più dimensioni. Se il file di dati è formato solamente dal numero di colonne necessario e non bisogna escludere nessuna colonna è possibile inizializzare il grafico semplicemente come 
\\\\
TGraph *nome = new TGraph (*filename) \\\\
\subsection{Costruttore di TGraphErrors}
TGraphErrors si inizializza alo stesso modo di tGraph ma bisogna aggiungere l'informazione sugli errori. Il primo metodo diventa:\\\\
TGraphErrors *nome = new TGraphErrors (numero\_punti, *nome\_arrayx, *nome\_arrayy, *nome\_errx, *nome\_erry)\\\\
dove err\_x ed err\_y sono array conteneti gli errori per ogni coordinata di ogni punto del grafico. Per specificare che non si vogliono inserire errori in una variabile, ad esempio la x, basta sostituire uno 0 a *nome\_errx\\\\
TGraphErrors *nome =new TGraphErrors(numero\_punti, *nome\_arrayx, *nome\_arrayy, 0, *nome\_erry);\\\\
Il secondo metodo invece diventa: \\\\
TGraphErrors *nome = new TGraphErrors (*filename, *format="\%lg \%lg \%lg") )\\\\
dove questa volta il numero minimo di colonne è 3. A rigore dovrebbe essere 4 (le prime due sono le misure di x ed y e le seconde i relativi errori) ma se sono solo 3 si interpreta l'ultima come gli errori sia sulla x che sulla y. Anche in questo caso se il file dei dati è formato solamente dalle colonne necessarie e non serve escludere colonne la sintassi si riduce a\\\\
TGraphErrors *nome = new TGraph (*filename)\\\\
ATTENZIONE: LA VIRGOLA NON \'{E} LETTA COME SEPARATORE DECIMALE, SERVE IL PUNTO. 
\subsection{Metodi per grafici}
Di seguito i principali metodi dei grafici
\begin{itemize}
	\item nome -> SetTitle(“Titolo; Titolo\_asseX; Titolo\_asseY");
	\item nome -> SetMarkerStyle(kOpenCircle);
	\item nome -> SetMarkerStyle(numero);
	\item nome -> SetMarkerColor(kBlue);
	\item nome -> SetLineWidth(2);
	\item nome -> SetLineColor(kBlue);
	\item nome -> GetPoint(i, x, y);
	prende l’iesimo punto del grafico
	\item nome -> GetX ()/GetY ();
	ritorna un puntatore ad un'array con tutte le coordinate x (o y). 
	\item nome -> GetN();
	ritorna il numero di punti nel grafico
	\item nome -> Integral();
	calcola l'integrale da inizio a fine grafico
	\item nome -> AddPoint(x,y);
	aggiunge un punto di coordinate x, y al grafico
	\item nome -> SetPoint (i,x,y);
	sovrascrive il valore dell'iesimo punto del grafico
	\item nome -> GetCorrelationFactor();
	calcola il coefficiente di correlazione lineare fra i punti
	\item nome -> GetCovariance();
	calcola la covarianza fra i punti
	\item nome -> Fit( "formula");
	effettua un fit a partire da una formula scritta come una stringa; deve essere scritta nella forma [0]+x*[1]
	\item nome -> Fit(nome\_funzione);
	effettua un fit a partire da una funzione precedentemente definita
\end{itemize}
(kOpenCirclee kBlue sono codici predefiniti in ROOT, li ho messi per esempio)

\section{Variabili globali}
I puntatori globali puntano ad informazioni generali sul file in uso sono principalmente 4:
\begin{itemize}
	\item gROOT\\
	informazione globale relativa alla sessione corrente con cui si può accedere a qualunque oggetto in essa
	\item gFile\\
	puntatore al root file corrente
	\item gStyle\\
	puntatore alle funzionalità di stile
	\item gRandom\\
	puntatore al generatore di numeri casuale, che verrà approfondito nella sezione \ref{sec:montecarlo}.
\end{itemize}
Di solito si crea una funzione prima di cominciare la funzione principale della macro in cui si settano le variabili globali, ad esempio
void SetStyle() \{
	gROOT->SetStyle("Plain");
	gStyle->SetPalette(57);
	gStyle->SetOptTitle(0);
\}
\section{Funzioni di ROOT}
Le funzioni sono oggetti che possono essere definite dall'utente o prese dagli oggetti di ROOT (ad esempio gaussiana, esponenziale, polinomiale,...). Le funzioni in generale sono di tipo TF seguito da un numero che ne indica la dimensione; ad esempio quelle ad una variabile sono contraddistinte dall'"1" alla fine del nome del tipo.
\subsection{Inizializzazione e funzioni utili}
Le funzioni definite dal'utente possono essere dichiarate in tre modi:
\begin{itemize}
	\item Stringa di caratteri nel costruttore\\\\
	TF1 *nome1 = new TF1("nome1", "sin(x)/x“, x\_iniziale, x\_finale);\\\\
	\'{E} inoltre possibile usare funzioni già esistenti\\\\
	TF1 *nome2 = new TF1("nome2", "nome1*x“, x\_iniziale, x\_finale);\\\\
	\item Stringa di caratteri con parametri da inizializzare\\\\
	TF1 *nome = new TF1 ("nome",“ [0]*x*sin([1]*x)",x\_iniziale, x\_finale);\\\\
	dove i numeri fra parentesi quadre sono parametri liberi che possono essere definiti con\\\\
	nome -> SetParameter(numero\_parametro, valore);\\\\
	dove "numero\_parametro" è il numero del parametro che vogliamo definire (nel nostro esempio o 0 o 1) e "valore" è il valore che vogliamo dare a quel parametro.\\
	\'{E} del tutto equivalente inizializzare la funzione specificando nel costruttore il numero di parametri:\\\\
	TF1 *nome = new TF1 ("nome","funzione",x\_iniziale, x\_finale, numero\_parametri);\\\\
	In questo caso sarà possibile definire tutti i parametri in una volta con\\\\
	nome -> SetParameters(valore1,valore2, ...);\\
	\item Definita in una macro\\
	Si usa la sintassi di C++, ad esempio:\\\\
	MyFunction(Double\_t *x, Double\_t *par)\{
		 Float\_txx =x[0];\\
		 Double\_t val= TMath::Abs(par[0]*sin(par[1]*xx)/xx);\\
		 return val;
		 \};\\
\end{itemize}
Esistono anche funzioni built-in ROOT che si dichiarano così:\\\\
TF1 *nome\_funz = new TF1("nome\_funz", "funzione");\\\\
dove "funzione" indica il nome della funzione di ROOT, alcuni esempi sono: linear, poln, gaus, expo... Ad esempio, la funzione gaus ha 3 parametri ed è definita come:\\\\
\ [0]*exp(-0.5 *(x-[1])/[2]) *(x-[1])/[2]) )\\\\
dove i parametri rappresentano:\\\\
\ [0] = ampiezza massima in corrispondenza della media\\
\ [1] = media\\
\ [2] = sigma\\\\
Alcune funzioni membro delle funzioni sono:
\begin{itemize}
	\item nome -> Eval(x);\\
	valuta la funzione in quel punto
	\item nome -> Integral(a, b)\\
	calcola l'integrale fra i punti a e b
\end{itemize}
\subsection{Fitting}
Per effettuare un fit di dati in un istogramma rispetto ad una funzione definita dall'utente bisogna:
\begin{enumerate}
	\item Creare una funzione con parametri liberi che pensiamo sia adeguata a fittare i dati\\\\
	TF1 *nome\_funz = new TF1("nome\_funz","funzione con parametri", x\_min, x\_max);\\
	\item Effettuare il fit, che non fa altro che assegnare parametri\\\\
	nome\_isto -> Fit("nome\_funz");\\
	\item Per accedere alla funzione di fit usiamo una funzione dedicata per prenderla:\\\\
	 TF1 *fit\_funz = h->GetFunction(“nome\_funz");\\\\
	 ora è una normale funzione; possiamo prendere i parametri con la funzione\\\\
	 fit\_funz -> GetParameter(i);\\
	 fit\_funz -> GetParameters(nome\_array);\\\\
	 il primo metodo prende solamente l'iesimo parametro mentre il secondo crea una array con tutti i parametri.\\
\end{enumerate}
Per effettuare un fit a partire da una funzione built-in ROOT basta scrivere:\\\\
nome\_isto -> Fit(funzione);\\\\
dove al posto di funzione si può mettere il nome di una funzione built-in ROOT (linear, poln, gaus, expo...)
Di seguito altre utili funzioni membro delle funzioni fit
\begin{itemize}
	\item fit\_funz -> GetChisquare(); 
	prende il Chiquadro
	\item fit\_funz -> GetNDF();
	prende i gradi di libertà
	\item fit\_funz -> GetParError(i);
	prende l'errore dell'iesimo parametro
	\item fit\_funz -> GetParErrors(nome\_array);
	crea un'array con tutti gli errori sui parametri. Si potrebbe anche creare un'oggetto array ed assegnargli la funzione con le parentesi vuote.
\end{itemize}
\'{E} possibile fittare anche grafici grazie all'overloading della funzione Fit:\\\\
nome\_grafico -> Fit("nome\_funzione");\\
 \subsubsection*{Fit in sotto-range}
 Spesso capita che si voglia fittare solo una parte di grafico o istogramma. Esistono vari modi per farlo:
 \begin{itemize}
 	\item Si dichiara un range di validità (dominio) nel costruttore della funzione e poi si fitta usando l'opzione "R":\\\\
 	TF1 *nome\_funz = new TF1("nome\_funz", "funzione", range\_min, range\_max);\\
 	nome\_isto -> Fit("nome\_funz", "R");\\
 	\item Specificare come opzione di Fit il range. In questo caso la funzione si dichiara nel modo classico e per fittare si scrive:\\\\
 	nome\_isto -> Fit("nome\_funz", " ", " ", range\_min, range\_max); 	
 \end{itemize}
Se si specifica il range è possibile fittare utilizzando più funzioni per i diversi range. Gli step da seguire sono:
\begin{enumerate}
	\item Definire le n funzioni con cui si vuole fittare (di solito si usano le built-in come "gaus" da mettere al posto di "fi") e fare la somma:\\\\
	TF1 *nome\_funz1 = new TF1(“nome\_funz1", f1, range1, range2);\\
	TF1 *nome\_funz2 = new TF1(“nome\_funz2", f2, range3, range4);\\
	TF1 *nome\_funzn = new TF1(“nome\_funz3", fn, rangen-1, rangen);\\
	//funzionesomma\\
	TF1 *total = new TF1("total","f1+f2+...+fn", range1, rangen);\\\\
	dove f1,...,fn sono le funzioni pre definite con cui si vuole fittare. 
	\item Fittare ogni sotto-range con le funzioni\\\\
	nome\_isto -> Fit(f1, "R");
	nome\_isto -> Fit(f2, "R");
	nome\_isto -> Fit(fn, "R");
	\item Recuperare i parametri dopo il fit ad esempio in una array\\\\
	Double\_t nome\_array[numero\_parametritot];\\
	g1->GetParameters(\&nome\_array[i]);\\
	g2->GetParameters(\&nome\_array[j]);\\
	g3->GetParameters(\&nome\_array[k]);\\\\
	dove la "\&" serve perchè si passa l'array by-reference e i numeri i,, j, k indicano i parametri che si devono prendere dalle singole funzioni. 
	\item Inserire i parametri risultanti in “total” (funzione somma)\\\\
	total -> SetParameters(nome\_array);\\
	\item Fittare nuovamente con la funzione somma per una stima definitiva dei parametri:\\\\
	nome\_isto -> Fit(total, "R")\\
	\item Estrarre risultati finali dai metodi delle funzioni fit già visti.
\end{enumerate}
\section{Legenda}
La legenda è un oggetto a se stante (TLegend) che deve essere disegnato sulla canvas. Un esempio è:\\\\
TLegend *nome = new TLegend(.1,.7,.3,.9,“Titolo");\\
nome -> AddEntry(oggetto1, "label1");\\
nome -> AddEntry(oggetto2, “label2");\\
nome -> Draw("SAME");\\\\
dove i numeri nella dichiarazione di TLegend sono parametri per le dimensioni del rettangolo della legenda; AddEntry aggiunge alla legenda oggetti (quindi non solo istogrammi o grafici ma anche numeri, ad esempio il Chiquadro). 
\section{Disegnare oggetti}
\subsection{Canvas}
Per prima cosa bisogna creare una "canvas" che è l'ambiente in cui è possibile disegnare gli oggetti. Un modo base per inizializzarla è\\\\
TCanvas *nome = new TCanvas();\\\\
In questo modo si crea una canvas vuota di dimensioni automatiche. \'{E} inoltre possibile specificare il nome, il titolo e le dimensioni:\\\\
TCanvas *nome = new TCanvas("nome", "titolo", larghezza, altezza);\\\\
\'{E} infine possibile dividere una canvas il più "pads" per accostare diversi grafici.\\\\
nome -> Divide(larghezza1, altezza1,larghezza2, altezza2);\\\\
oppure, se si volesse dividere in 4 parti uguali:\\\\
nome -> Divide(2, 2);\\\\
Infine, è possibie salvare il la canvas come immagine con il seguente comando\\\\
nomecanvas -> Print("nomefile.C/.png/.gif/.pdf/.jpg...")
\subsection{Draw()}
Una volta creata la canvas, per disegnare gli istogrammi si usa il metodo Draw(). Di base è possibile disegnare un oggetto semplicemente con\\\\
nome -> Draw();\\\\ 
Esistono però molte opzioni di questa funzione, riportate in appendice \ref{sec:Draw}. Le opzioni si inseriscono con\\\\
nome -> Draw("lettere")\\\\
Le opzioni si possono unire con virgole o semplicemente scrivendo le lettere di seguito.
\section{Generazione di Montecarlo}\label{sec:montecarlo}
I generatori di numeri casuali (PRNG) sono di fondamentale importanza in fisica per studiare eventi complessi formati da numerose acquisizioni dati.
Un PRNG può seguire una qualsiasi p.d.f.; in ROOT è possibile generare numeri casuali che seguono p.d.f. built-in ROOT o user-defined. Alcuni esempi di p.d.f. built-in ROOT sono: 
\begin{itemize}
	\item Uniform(x1, x2)
	vuole la x iniziale e finale fra cui generare
	\item Gaus(media, sigma)
	vuole media e deviazione standard
	\item Poisson(media)
	vuole la media
	\item Binomial(ntot, prob)
	vuole il numero totale di elementi e la probabilità che avvenga l'evento binario.
	\item Exp(tau)
	vuole il coefficiente che sta all'esponente
	\item Integr(imax)	
	\item Landau(moda, sigma)
	vuole moda e deviazione standard
\end{itemize}
Un esempio di generazione casuale apartire da una built-in, utilizzando il metodo gRandom è:\\\\
Double\_t x = gRandom -> Uniform(xmin, xmax);\\\\
\'{E} inoltre possibile generare numeri casuali partendo da una funzione user-defined utilizzando il metodo GetRandom():\\\\
TF1 *nome\_funz = new TF1("nome\_funz","funzione\_userdef",x\_min, x\_max);\\
doubler = nome\_funz -> GetRandom();\\\\
\\\'{E} anche possibile riempire direttamente un'istogramma di entrate casuali con il metodo FillRandom():\\\\
 nome\_isto -> FillRandom(“funzione”, num\_entrate);\\\\
dove "funzione può essere sa user-defined che built-in.
\newpage
\appendix
\section{Opzioni di Draw}\label{sec:Draw}
\begin{figure}[h!]
	\centering
	\includegraphics[width=0.7\linewidth]{"Screenshot 2022-06-04 221503"}
\end{figure}
\FloatBarrier
\begin{figure}[h!]
	\centering
	\includegraphics[width=0.7\linewidth]{"Screenshot 2022-06-04 222020"}
\end{figure}
\FloatBarrier
\begin{figure}[h!]
	\centering
	\includegraphics[width=0.7\linewidth]{"Screenshot 2022-06-04 222115"}
\end{figure}
\FloatBarrier
\newpage
\section{Esempi}
\begin{figure}[h!]
	\centering
	\includegraphics[width=0.8\linewidth]{"Screenshot 2022-06-06 212045"}
	\caption{Programma che verifica che una binomiale tende ad una gaussiana usando la generazione di Montecarlo.}
	\label{fig:screenshot-2022-06-06-212045}
\end{figure}
\begin{figure}[h!]
	\centering
	\includegraphics[width=0.8\linewidth]{"Screenshot 2022-06-06 212607"}
	\caption{Programma che effettua il plot di dati con errore su una delle due misure usando TGraphErrors.}
	\label{fig:screenshot-2022-06-06-212607}
\end{figure}
\begin{figure}[h!]
	\centering
	\includegraphics[width=0.8\linewidth]{"Screenshot 2022-06-06 212449"}
	\caption{Programma che crea due istogrammi con i tempi di caduta del pendolo di Maxwell prendendo dati da un file esterno usando ifstream.}
	\label{fig:screenshot-2022-06-06-212449}
\end{figure}

\FloatBarrier
\end{document}